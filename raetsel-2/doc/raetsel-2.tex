%coding:utf-8
\documentclass[10pt, fleqn]{article}

\usepackage{layout}

\title{Lösung Servish Rätsel 2 -- Einkaufen mit Samantha}
\author{Daniel Winz}

\begin{document}
\maketitle
\newpage
\tableofcontents
\newpage

\section{Aufgabenstellung}

\subsection*{SERVISH RÄTSEL II -- Einkaufen mit Samantha}

\subsubsection*{Turniermodus}
Der Gewinner erhält 50\% des Betrages, der Zweitbeste 30\% und der drittbeste 20\%. 

\subsubsection*{Bewerbung möglich bis}
26. Januar 2014 \\
20:00 Uhr 

\subsubsection*{Rücktritt möglich bis}
28. Januar 2014 \\
20:00 Uhr 

\subsubsection*{Projektabschluss}
30. Januar 2014 \\
20:00 Uhr 

\subsubsection*{Rubrik}
Lifestyle \& Unterhaltung / Andere

\subsubsection*{Zusammenfassung}
Nachdem das erste, anspruchsvolle Rätsel erfolgreich gelöst wurde, präsentiert 
euch Sam Servish ein neues Rätsel. 

\subsubsection*{Beschreibung}
Samantha Servish geht in einem Laden einkaufen. An der Kasse angekommen, legt 
sie ihre vier Produkte aufs Band. "Das macht 7 Franken und 77 Rappen", sagt 
die Verkäuferin. \\
"'Ach nein, jetzt habe ich die Preise versehentlich multipliziert!"'. \\
Nochmals tippt sie die vier Produkte und zählt diese nun zusammen. \\
"Was für ein Zufall, das macht wieder 7 Franken und 77 Rappen!". \\\\
%
Wieviel kosten die einzelnen Produkte? \\\\
%
(ein kleiner Tipp: Ein Rappen ist die kleinste Zahlungseinheit) \\\\
%
Viel Spass beim Lösen! \\
Euer Sam Servish \\\\
%
Bewerbt euch für dieses "Rätsel" und reicht eure Lösung erst ein, nachdem Sam 
euch beauftragt hat. Ansonsten ist eure Lösung für alle anderen Teilnehmer 
ersichtlich. Der Auftrag wird im Turniermodus durchgeführt. 

\subsubsection*{Projektowner}
Sam Servish 

\subsubsection*{Eingetragen am}
16. Januar 2014 \\
(vor 8 Tagen) 

\newpage

\section{Lösungsansätze}
Aus der Aufgabenstellung sind vier Variablen gesucht, aber nur zwei Gleichungen 
gegeben: 
\[ a + b + c + d = 7.77 \]
\[ a \cdot b \cdot c \cdot d = 7.77 \]
\[ 100 \cdot a \in \mathbb{N} \qquad 100 \cdot b \in \mathbb{N} \qquad 
100 \cdot c \in \mathbb{N} \qquad 100 \cdot d \in \mathbb{N} \]
Die Lösung wird mit dem Rechner gesucht. Der dazu benutzte Algorithmus wird 
dabei Schritt für Schritt verbessert. Als Programmiersprache wird Python 
verwendet. Um die Berechnungen mit Ganzzahlen durchführen zu können, werden 
die Preise in Rappen berechnet. Das führt jedoch dazu, dass die Multiplikation 
der Zahlen einen um den Faktor $10^8$ grösseren Wert $(777000000)$ ergibt, da 
jeder Faktor mit $100$ multipliziert ist. 
\[ i + j + k + l = 777 \]
\[ i \cdot j \cdot k \cdot l = 777000000 \]
\[ i \in \mathbb{N} \qquad j \in \mathbb{N} \qquad 
k \in \mathbb{N} \qquad l \in \mathbb{N} \]

\subsection{Bruteforce}
Zunächst werden für alle vier Produkte alle Möglichkeiten ausprobiert. Die 
Variablen \verb!i!, \verb!j!, \verb!k! und \verb!l! entsprechen dabei den 
Preisen der einzelnen Produkte. \\
Mit den vier \verb!for! Schleifen werden alle vier Variablen durchiteriert. 
Mit den \verb!if! Verzweigungen wird zuerst das Produkt und anschliessend die 
Summe der Variablen überprüft. Wenn beide korrekt sind, wird das Resultat 
mit der Funktion \verb!printresult! ausgegeben. 
% \lstinputlisting{../source/raetsel_1.py}

\subsection{Ein bisschen Algebra}
Bei obigem Algorithmus wird jede Kombination überprüft. Mit den obigen 
Gleichungen kann jedoch eine der Variablen eliminiert werden. 
\[ l = 777 - i - j - k \]
\[ i \cdot j \cdot k \cdot (777 - i - j - k) = 777000000 \]
Damit kann eine der Schleifen weggelassen und \verb!l! direkt berechnet werden. 
% \lstinputlisting{../source/raetsel_2.py}

\subsection{Grenzen sinnvoll nutzen}
Die kleinste der Zahlen kann höchstens ein Viertel der Summe sein. Daher muss 
eine der Variabel nur bis da hin iteriert werden. Von den übrigen Zahlen muss 
die Kleinste kleiner als ein Drittel der Summe sein. Von den übrigen beiden 
Zahlen darf die Kleinere höchstens halb so gross wie die Hälfte Sein. Mit 
diesen Regeln kann der Algorithmus weiter optimiert werden. 
% \lstinputlisting{../source/raetsel_3.py}

\subsection{Teiler}
Der grösste Teiler der Zahl $777000000$ ist $37$. Eine der Variablen kann nun 
in Schritten von 37 iteriert werden. 
% \lstinputlisting{../source/raetsel_4.py}

\subsection{Primzahlzerlegung}
Ein komplett anderer Ansatz basiert auf der Primzahlzerlegung. Die Zahl 
$777000000$ ist aus den folgenden Primzahlen zusammengesetzt: 
\[ 777000000 = 2 \cdot 2 \cdot 2 \cdot 2 \cdot 2 \cdot 2 \cdot 3 
\cdot 5 \cdot 5 \cdot 5 \cdot 5 \cdot 5 \cdot 5 \cdot 7 \cdot 35 
= 2^6 \cdot 3^1 \cdot 5^6 \cdot 7^1 \cdot 35^1 \]
Die einzelnen Beträge müssen aus diesen Primzahlen zusammengesetzt sein. \\
Für jede Primzahl wird nun eine Liste erzeugt. Die Liste beinhaltet vier Werte. 
Jeder Wert ist dabei einem der vier Beträge zugeordnet. Der Wert in der Liste 
entspricht dabei der Potenz der jeweiligen Primzahl beim entsprechenden Betrag. 
\[ \begin{pmatrix}\text{Betrag 1}\\\text{Betrag 2}
\\\text{Betrag 3}\\\text{Betrag 4}\end{pmatrix} \]
Für Primzahlen, die nur einmal auftreten, wird eine Liste mit folgenden Werten 
erzeugt: 
\[ \begin{pmatrix}1\\0\\0\\0\end{pmatrix} \qquad 
   \begin{pmatrix}0\\1\\0\\0\end{pmatrix} \qquad 
   \begin{pmatrix}0\\0\\1\\0\end{pmatrix} \qquad 
   \begin{pmatrix}0\\0\\0\\1\end{pmatrix} \qquad \]
Für die anderen Primzahlen, die sechsmal auftreten wird eine zweite Liste 
erzeugt. Dazu werden alle Kombinationen mit Zahlen von 0 bis 6 ausprobiert. 
Diejenigen, die in Summe $6$ ergeben werden in der Liste gespeichert. 
\[ \begin{pmatrix}0\\0\\0\\6\end{pmatrix} \qquad 
   \begin{pmatrix}0\\0\\1\\5\end{pmatrix} \qquad 
   \begin{pmatrix}0\\0\\2\\4\end{pmatrix} \qquad 
   \begin{pmatrix}0\\0\\3\\3\end{pmatrix} \qquad 
   \begin{pmatrix}0\\0\\4\\2\end{pmatrix} \qquad 
   \begin{pmatrix}0\\0\\5\\1\end{pmatrix} \qquad 
   \begin{pmatrix}0\\0\\6\\0\end{pmatrix} \qquad 
   \begin{pmatrix}0\\1\\0\\5\end{pmatrix} \qquad 
   \begin{pmatrix}0\\1\\1\\4\end{pmatrix} \qquad 
   \dots \qquad
   \begin{pmatrix}6\\0\\0\\0\end{pmatrix} \qquad \]
Durch die nun erzeugten Listen wird nun mit jeder Primzahl durchiteriert. Jede 
korrekte Lösung wird ausgegeben. 
% \lstinputlisting{../source/raetsel_5.py}

\subsection{Primzahlzerlegung mit Festlegung der grössten Primzahl}
Eine Optimierung des obigen Algorithmus liegt darin, eine der Primzahlen auf 
einen Betrag festzulegen. Damit kann eine Schleife gespart werden. 
% \lstinputlisting{../source/raetsel_6.py}

\subsection{Primzahlzerlegung mit optimierten Listen}
Bei der Generierung der Listen für mehrfach vorhandene Primzahlen sind viele 
Zahlenkombinationen mehrfach vorhanden. Die kann durch die Abgrenzung der 
Bereiche für die einzelnen Zahlenwerte in der Liste erfolgen. Wichtig ist 
dabei, dass diese Optimierung nur bei einer Primzahl durchgeführt wird, da 
ansonsten möglicherweise keine Lösung gefunden wird. 
% \lstinputlisting{../source/raetsel_7.py}

\subsection{Primzahlzerlegung mit manuell erstellten Listen}
Die Liste kann auf eine kleine Anzahl von Kombinationen heruntergebrochen 
werden: 
\[ \begin{pmatrix}6\\0\\0\\0\end{pmatrix} \qquad 
   \begin{pmatrix}5\\1\\0\\0\end{pmatrix} \qquad 
   \begin{pmatrix}4\\2\\0\\0\end{pmatrix} \qquad 
   \begin{pmatrix}4\\1\\1\\0\end{pmatrix} \qquad 
   \begin{pmatrix}3\\3\\0\\0\end{pmatrix} \qquad 
   \begin{pmatrix}3\\2\\1\\0\end{pmatrix} \qquad 
   \begin{pmatrix}3\\1\\1\\1\end{pmatrix} \qquad 
   \begin{pmatrix}2\\2\\2\\0\end{pmatrix} \qquad 
   \begin{pmatrix}2\\2\\1\\1\end{pmatrix} \qquad \]
Diese Liste wird nun manuell eingegeben. 
% \lstinputlisting{../source/raetsel_8.py}

\section{Resultate}
Das Resultat aller Algorithmen sind die Zahlen \verb!80!, \verb!125!, 
\verb!222! und \verb!350! in verschiedenen Reihenfolgen. 
Die einzelnen Produkte kosten also CHF 0.80, CHF 1.25, CHF 2.22 und CHF 3.50. 
Sowohl die Summe als auch das Produkt dieser Werte ergibt 7.77 CHF. 
\[ \begin{pmatrix}80\\125\\222\\350\end{pmatrix} \]

\subsection{Laufzeit der Algorithmen}
\begin{tabular}{@{}lc}
\rowcolor{white}\textbf{Algorithmus}                                    
  &\textbf{Zeit $[s]$} \\
\rowcolor{lgray}Bruteforce                                              
  & $145028$ \\
\rowcolor{white}Ein bisschen Algebra                                    
  & $239.8$ \\
\rowcolor{lgray}Grenzen sinnvoll nutzen                                 
  & $7.01$ \\
\rowcolor{white}Teiler                                                  
  & $0.428$ \\
\rowcolor{lgray}Primzahlzerlegung                                       
  & $1.01$ \\
\rowcolor{white}Primzahlzerlegung mit Festlegung der grössten Primzahl  
  & $0.216$ \\
\rowcolor{lgray}Primzahlzerlegung mit optimierten Listen                
  & $0.298$ \\
\rowcolor{white}Primzahlzerlegung mit manuell erstellten Listen         
  & $0.109$ \\

\end{tabular}

\clearpage

\begin{appendices}
  \pagenumbering{roman}

  \section{Listings}
  \subsection{Bruteforce}
  \lstinputlisting{../source/raetsel_1.py}
  \subsection{Ein bisschen Algebra}
  \lstinputlisting{../source/raetsel_2.py}
  \subsection{Grenzen sinnvoll nutzen}
  \lstinputlisting{../source/raetsel_3.py}
  \subsection{Teiler}
  \lstinputlisting{../source/raetsel_4.py}
  \subsection{Primzahlzerlegung}
  \lstinputlisting{../source/raetsel_5.py}
  \subsection{Primzahlzerlegung mit Festlegung der grössten Primzahl}
  \lstinputlisting{../source/raetsel_6.py}
  \subsection{Primzahlzerlegung mit optimierten Listen}
  \lstinputlisting{../source/raetsel_7.py}
  \subsection{Primzahlzerlegung mit manuell erstellten Listen}
  \lstinputlisting{../source/raetsel_8.py}  

  \section{Resultate}
  \subsection{Bruteforce}
  \lstinputlisting{../source/result_1.txt}
  \subsection{Ein bisschen Algebra}
  \lstinputlisting{../source/result_2.txt}
  \subsection{Grenzen sinnvoll nutzen}
  \lstinputlisting{../source/result_3.txt}
  \subsection{Teiler}
  \lstinputlisting{../source/result_4.txt}
  \subsection{Primzahlzerlegung}
  \lstinputlisting{../source/result_5.txt}
  \subsection{Primzahlzerlegung mit Festlegung der grössten Primzahl}
  \lstinputlisting{../source/result_6.txt}
  \subsection{Primzahlzerlegung mit optimierten Listen}
  \lstinputlisting{../source/result_7.txt}
  \subsection{Primzahlzerlegung mit manuell erstellten Listen}
  \lstinputlisting{../source/result_8.txt}  

\end{appendices}

\end{document}
