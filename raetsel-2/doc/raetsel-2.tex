%coding:utf-8
\documentclass[10pt, fleqn]{article}

\usepackage{layout}

\title{Lösung Servish Rätsel 2 -- }
\author{Daniel Winz}

\begin{document}
\vfill
\maketitle
\vfill
\tableofcontents
\vfill
\newpage

\section{Aufgabenstellung}

\subsection*{SERVISH RÄTSEL II}

\subsubsection*{Turniermodus}
Der Gewinner erhält 50\% des Betrages, der Zweitbeste 30\% und der drittbeste 20\%. 

\subsubsection*{Bewerbung möglich bis}

\subsubsection*{Rücktritt möglich bis}

\subsubsection*{Projektabschluss}

\subsubsection*{Rubrik}

\subsubsection*{Zusammenfassung}

\subsubsection*{Beschreibung}

\subsubsection*{Projektowner}

\subsubsection*{Eingetragen am}

\newpage

\section{Lösungsansätze}
Die Lösung wird mit dem Rechner gesucht. Der dazu benutzte Algorithmus wird 
dabei Schritt für Schritt verbessert. Als Programmiersprache wird Python 
verwendet. 

\subsection{Bruteforce}

\subsection{Ein bisschen Algebra}

\subsection{Grenzen sinnvoll setzen}

\subsection{Teiler}

\subsection{Primzahlzerlegung}

\end{document}
