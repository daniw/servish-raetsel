%coding:utf-8
\documentclass[10pt, fleqn]{article}

\usepackage{layout}

\title{Lösung Servish Rätsel 2 -- Einkaufen mit Samantha}
\author{Daniel Winz}

\begin{document}
\vfill
\maketitle
\vfill
\tableofcontents
\vfill
\newpage

\section{Aufgabenstellung}

\subsection*{SERVISH RÄTSEL II -- Einkaufen mit Samantha}

\subsubsection*{Turniermodus}
Der Gewinner erhält 50\% des Betrages, der Zweitbeste 30\% und der drittbeste 20\%. 

\subsubsection*{Bewerbung möglich bis}
26. Januar 2014 \\
20:00 Uhr 

\subsubsection*{Rücktritt möglich bis}
28. Januar 2014 \\
20:00 Uhr 

\subsubsection*{Projektabschluss}
30. Januar 2014 \\
20:00 Uhr 

\subsubsection*{Rubrik}
Lifestyle \& Unterhaltung / Andere

\subsubsection*{Zusammenfassung}
Nachdem das erste, anspruchsvolle Rätsel erfolgreich gelöst wurde, präsentiert 
euch Sam Servish ein neues Rätsel. 

\subsubsection*{Beschreibung}
Samantha Servish geht in einem Laden einkaufen. An der Kasse angekommen, legt 
sie ihre vier Produkte aufs Band. "Das macht 7 Franken und 77 Rappen", sagt 
die Verkäuferin. \\
"'Ach nein, jetzt habe ich die Preise versehentlich multipliziert!"'. \\
Nochmals tippt sie die vier Produkte und zählt diese nun zusammen. \\
"Was für ein Zufall, das macht wieder 7 Franken und 77 Rappen!". \\\\
%
Wieviel kosten die einzelnen Produkte? \\\\
%
(ein kleiner Tipp: Ein Rappen ist die kleinste Zahlungseinheit) \\\\
%
Viel Spass beim Lösen! \\
Euer Sam Servish \\\\
%
Bewerbt euch für dieses "Rätsel" und reicht eure Lösung erst ein, nachdem Sam 
euch beauftragt hat. Ansonsten ist eure Lösung für alle anderen Teilnehmer 
ersichtlich. Der Auftrag wird im Turniermodus durchgeführt. 

\subsubsection*{Projektowner}
Sam Servish 

\subsubsection*{Eingetragen am}
16. Januar 2014 \\
(vor 8 Tagen) 

\newpage

\section{Lösungsansätze}
Die Lösung wird mit dem Rechner gesucht. Der dazu benutzte Algorithmus wird 
dabei Schritt für Schritt verbessert. Als Programmiersprache wird Python 
verwendet. 

\subsection{Bruteforce}

\subsection{Ein bisschen Algebra}

\subsection{Grenzen sinnvoll setzen}

\subsection{Teiler}

\subsection{Primzahlzerlegung}

\end{document}
