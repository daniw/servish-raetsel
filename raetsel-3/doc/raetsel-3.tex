%coding:utf-8
\documentclass[a4paper, 10pt, fleqn]{article}

\usepackage{layout}

\title{Lösung Servish Rätsel 3 -- Stan gesellt sich dazu}
\author{daniw}

\begin{document}
\maketitle
\newpage
\tableofcontents
\newpage

\section{Aufgabenstellung}

\subsection*{SERVISH RÄTSEL III - Stan gesellt sich dazu}

\subsubsection*{Turniermodus}
Der Gewinner erhält 50\% des Betrages, der Zweitbeste 30\% und der drittbeste 20\%. 

\subsubsection*{Bewerbung möglich bis}
12. Februar 2014, 
22:00 Uhr 

\subsubsection*{Rücktritt möglich bis}
14. Februar 2014, 
22:00 Uhr 

\subsubsection*{Projektabschluss}
16. Februar 2014, 
22:00 Uhr 

\subsubsection*{Rubrik}
Lifestyle \& Unterhaltung / Andere

\subsubsection*{Zusammenfassung}
Wir wünschen viel Spass mit dem dritten Servish Rätsel

\subsubsection*{Beschreibung}
Sam, Samantha und Stan Servish (der tennisspielende Onkel) haben die Aufgabe 
zwei Zahlen unter den folgenden Bedingungen zu finden:
\begin{itemize}
\item die Zahlen sind ganzzahlig liegen zwischen 1 und 1000
\item es könnten auch beide Zahlen gleich sein
\item Samantha erfährt die Summe
\item Stan kennt die Differenz
\item und Sam weiss das Produkt der beiden Zahlen
\end{itemize}
Anschliessend entwickelt sich das folgende Gespräch:
\\\\
\begin{tabular}{ll}
Sam:      & "`Ich weiss die beiden Zahlen nicht. "' \\
Samantha: & "`Das wusste ich doch schon. "' \\
Sam:      & "`Ich kenne die Zahlen jetzt. "' \\
Samantha: & "`Ich auch. "' \\
Stan:     & "`Ich noch nicht. Eine Zahl kann ich vermuten. Sicher bin ich mir 
              aber nicht. "' \\
Sam:      & "`Deine Vermutung ist falsch. "' \\
Stan:     & "`Jetzt kenne ich sie auch. "' \\
\end{tabular}
\\\\
Welches sind die beiden Zahlen?
\\\\
Viel Spass beim Lösen!
\\\\
Euer Sam Servish

\subsubsection*{Projektowner}
Sam Servish 

\subsubsection*{Eingetragen am}
31. Januar 2014 \\

\newpage

\section{Idee des Lösungsverfahrens}
Das folgende Lösungsverfahren beruht auf dem Ausschlussprinzip. Es wird eine 
Liste mit allen möglichen Kombinationen erzeugt, welche Schritt für Schritt 
reduziert wird. \\
Siehe auch: \\
\url{http://mchouza.wordpress.com/2011/05/31/%
solving-the-product-sum-difference-riddle/}. 

\section{Lösungsverfahren Schritt für Schritt}
Aus der Aufgabenstellung ist folgendes gegeben: 
\begin{itemize}
\item die Zahlen sind ganzzahlig liegen zwischen 1 und 1000
\item es könnten auch beide Zahlen gleich sein
\item Samantha erfährt die Summe,
\item Stan kennt die Differenz,
\item und Sam weiss das Produkt der beiden Zahlen
\end{itemize}

Als Erstes wird eine Liste mit allen möglichen Kombinationen generiert. Dabei 
ist die zweite Zahl immer grösser oder gleich der ersten Zahl. Damit werden 
Kombinationen, welche durch das Vertauschen der Zahlen einer anderen 
Kombination entstehen nur einmal berücksichtigt. 
\lstinputlisting[firstline=1, lastline=8, firstnumber=1]{../source/raetsel-3.py}

\textbf{Sam: }\emph{Ich weiss die beiden Zahlen nicht. } \\
Da Sam die Zahlen nicht kennt, muss es mehrere Kombinationen geben, die das 
Produkt ergeben, das Sam hat. \\
Dazu wird eine Liste für alle Produkte generiert. Anschliessend wird gezählt, 
wie oft jedes Produkt in den Kombinationen vorkommt. Es kommen nun nur noch die 
Kombinationen in Frage, bei welchen mehrere Kombinationen das selbe Produkt 
haben. 
\lstinputlisting[firstline=9, lastline=21, firstnumber=9]{../source/raetsel-3.py}

\textbf{Samantha: }\emph{Das wusste ich doch schon. } \\
Da Samantha wusste, dass Sam keine der Zahlen kannte, muss sie eine Summe 
erhalten haben, zu der es nur Kombinationen gibt, die Produkte geben, zu denen 
es mehrere Kombinationen gibt. Also kann Samantha zu jeder mit ihrer Summe 
möglichen Kombination prüfen, ob es nur eine Kombination mit dem selben Produkt 
gibt. 
\lstinputlisting[firstline=22, lastline=35, firstnumber=22]{../source/raetsel-3.py}

\textbf{Sam: }\emph{Ich kenne die Zahlen jetzt. } \\
Jetzt muss die Kombination eine der verbleibenden Kombinationen sein, zu dessen 
Produkt es keine zweite Kombination gibt. 
\lstinputlisting[firstline=36, lastline=48, firstnumber=36]{../source/raetsel-3.py}

\textbf{Samantha: }\emph{Ich auch. } \\
Zudem darf es zur Summe des Produkts nur eine Kombination geben. 
\lstinputlisting[firstline=49, lastline=61, firstnumber=49]{../source/raetsel-3.py}

\textbf{Stan: }\emph{Ich noch nicht. Eine Zahl kann ich vermuten. 
Sicher bin ich mir aber nicht.} \\
Stan kennt die Zahlen nicht. Das bedeutet, dass es zu seiner Differenz mehrere 
Kombinationen gibt. Da er eine der Zahlen vermuten kann, kommt diese 
anscheinend häufig vor. Da er die Zahl vermuten kann, sie jedoch nicht kennt, 
gibt es noch mindestens eine weitere Kombination, in welcher die Zahl nicht 
vorkommt. \\
Daher können alle Kombinationen ausgeschlossen werden, die Differenzen ergeben, 
die weniger als drei mal vorkommen. Nun werden alle noch vorkommenden 
Differenzen in der Liste \verb?diffs? gespeichert. Anschliessend werden in der 
Liste \verb?fdiff? alle Differenzen und die zugehörigen Kombinationen 
gespeichert. Mehrfach vorhandene Zahlen werden nun in die Liste 
\verb?gelements? gespeichert. Anschliessend werden Differenzen, in denen es 
nicht mehrfach vorhandene Zahlen gibt verworfen. 
\lstinputlisting[firstline=62, lastline=104, firstnumber=62]{../source/raetsel-3.py}

\textbf{Sam: }\emph{Deine Vermutung ist falsch. } \\
Nun werden alle Kombinationen mit mehrfach vorhandenen Zahlen verworfen. 
\lstinputlisting[firstline=105, lastline=110, firstnumber=105]{../source/raetsel-3.py}

\textbf{Stan: }\emph{Jetzt kenne ich sie auch. }
Da jetzt nur noch eine Kombination übrig ist, kann diese ausgegeben werden. 
Falls zu diesem Zeitpunkt noch mehrere Zahlen übrig wären, könnten diese wieder 
über die Anzahl Kombinationen für jede Differenz aussortiert werden. 
\lstinputlisting[firstline=111, lastline=115, firstnumber=111]{../source/raetsel-3.py}

\section{Lösung}
Die Rückgabe aus obigem Python Skript lautet: 
\begin{lstlisting}
[(64, 73)]
\end{lstlisting}
Somit sind 64 und 73 die beiden Zahlen. 

\clearpage

\begin{appendices}
  \pagenumbering{roman}

  \section{Listing}
  \lstinputlisting{../source/raetsel-3.py}

\end{appendices}

\end{document}
