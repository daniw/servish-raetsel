%coding:utf-8
\documentclass[a4paper, 10pt, fleqn]{article}

\usepackage{layout}

\title{Lösung Servish Rätsel 3 -- Stan gesellt sich dazu}
\author{daniw}

\begin{document}
\maketitle
\newpage
\tableofcontents
\newpage

\section{Aufgabenstellung}

\subsection*{SERVISH RÄTSEL III - Stan gesellt sich dazu}

\subsubsection*{Turniermodus}
Der Gewinner erhält 50\% des Betrages, der Zweitbeste 30\% und der drittbeste 20\%. 

\subsubsection*{Bewerbung möglich bis}
12. Februar 2014, 
22:00 Uhr 

\subsubsection*{Rücktritt möglich bis}
14. Februar 2014, 
22:00 Uhr 

\subsubsection*{Projektabschluss}
16. Februar 2014, 
22:00 Uhr 

\subsubsection*{Rubrik}
Lifestyle \& Unterhaltung / Andere

\subsubsection*{Zusammenfassung}
Wir wünschen viel Spass mit dem dritten Servish Rätsel

\subsubsection*{Beschreibung}
Sam, Samantha und Stan Servish (der tennisspielende Onkel) haben die Aufgabe 
zwei Zahlen unter den folgenden Bedingungen zu finden:
\begin{itemize}
\item die Zahlen sind ganzzahlig liegen zwischen 1 und 1000
\item es könnten auch beide Zahlen gleich sein
\item Samantha erfährt die Summe,
\item Stan kennt die Differenz,
\item und Sam weiss das Produkt der beiden Zahlen
\end{itemize}
Anschliessend entwickelt sich das folgende Gespräch:
\\\\
\begin{tabular}{ll}
Sam:      & "`Ich weiss die beiden Zahlen nicht."' \\
Samantha: & "`Das wusste ich doch schon."' \\
Sam:      & "`Ich kenne die Zahlen jetzt."' \\
Samantha: & "`Ich auch."' \\
Stan:     & "`Ich noch nicht. Eine Zahl kann ich vermuten. Sicher bin ich mir aber nicht."' \\
Sam:      & "`Deine Vermutung ist falsch."' \\
Stan:     & "`Jetzt kenne ich sie auch."' \\
\end{tabular}
\\\\
Welches sind die beiden Zahlen?
\\\\
Viel Spass beim Lösen!
\\\\
Euer Sam Servish

\subsubsection*{Projektowner}
Sam Servish 

\subsubsection*{Eingetragen am}
31. Januar 2014 \\

\newpage


\end{document}
