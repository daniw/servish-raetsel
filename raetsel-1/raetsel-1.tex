%coding:utf-8
\documentclass[10pt, fleqn]{article}

\usepackage{layout}

\title{Lösung Servish Rätsel 1 -- Sam Servish im Fahrstuhl}
\author{daniw}

\begin{document}
\vfill
\maketitle
\vfill
\tableofcontents
\vfill
\newpage

\section{Aufgabenstellung}
\subsection*{SERVISH RÄTSEL}
\subsubsection*{Turniermodus}
Der Gewinner erhält 50\% des Betrages, der Zweitbeste 30\% und der drittbeste 20\%. 

\subsubsection*{Bewerbung möglich bis}
12. Januar 2014 \\
17:00 Uhr

\subsubsection*{Rücktritt möglich bis}
13. Januar 2014 \\
17:00 Uhr

\subsubsection*{Projektabschluss}
15. Januar 2014 \\
17:00 Uhr

\subsubsection*{Rubrik}
Lifestyle \& Unterhaltung / Andere

\subsubsection*{Zusammenfassung}
Für alle die gerne Rätsel lösen habe ich genau das Richtige! Ab sofort stelle 
ich euch alle zwei Wochen ein neues kniffliges Rätsel aus den 
unterschiedlichsten Bereichen. Mit diesem Beitrag gehts ab heute los!

\subsubsection*{Beschreibung}
Löst das Rätsel und sichert euch die Möglichkeit auf 400 Servish Points. Die 
könnt ihr im Shop gegen tolle Prämien eintauschen.

Bewerbt euch für dieses "'Rätsel"' und reicht eure Lösung erst ein, nachdem Sam 
euch beauftragt hat. Ansonsten ist eure Lösung für alle anderen Teilnehmer 
ersichtlich. Der Auftrag wird im Turniermodus durchgeführt. Informationen dazu 
findet ihr im Hilfebereich.

Und nun zum Rätsel!

Zeiterfassung im Fahrstuhl
Sam Servish arbeitet in einem Appartementhaus und bedient dort den Fahrstuhl. 
Da es sich um ein sehr altes Haus handelt und der Fahrstuhl niemals erneuert 
wurde, muss Sam seine Arbeitszeit mit einer alten Stechuhr im Inneren des 
Fahrstuhls registrieren. Diese Stechuhr besteht aus einer alten Uhr mit einem 
Pendel. Egal ob der Fahrstuhl nach oben oder nach unten fährt, die 
Beschleunigungen des Fahrstuhls sind völlig identisch (kleiner als die 
Gravitationsbeschleunigung g), gleichmässig und konstant. Wenn Sam im 
Stundenlohn angestellt ist, erhält er dann zu viel oder zu wenig?

Viel Spass beim Lösen!
Euer Sam Servish

\subsubsection*{Projektowner}
Sam Servish 

\subsubsection*{Eingetragen am}
03. Januar 2014 

\newpage

\section{Theoretische Betrachtung}
Die Schwingungsfrequenz eines Pendels kann mit folgender Formel berechnet 
werden: 
\[ f_1 = \frac{1}{2 \pi} \cdot \sqrt{\frac{m \cdot g \cdot d}{I_s}} \]
Da in diesem Fall $g$ variiert kann obige Formel wie folgt ergänzt werden: 
\[ f_2 
= \frac{1}{2 \pi} \cdot \sqrt{\frac{m \cdot (g + \Delta g) \cdot d}{I_s}} \]
Die mittlere Frequenz während des Anfahrens und des Abbremsens ist dann: 
\[ f_2 
= \frac{1}{2 \pi} \cdot \frac{\sqrt{\frac{m \cdot (g + \Delta g) \cdot d}{I_s}} 
+ \sqrt{\frac{m \cdot (g - \Delta g) \cdot d}{I_s}}}{2} \]
\[ f_2 
= \frac{1}{4 \pi} \cdot \sqrt{\frac{m \cdot d}{I_s}} \cdot 
\left(\sqrt{g + \Delta g} + \sqrt{g - \Delta g}\right) \]
Für die Entscheidung, ob Sam Servish zuviel oder zu wenig erhält, ist nur die 
Differenz einer nicht bewegten Uhr und der Uhr im Fahrstuhl relevant. 
\[ \frac{f_2}{f_1} 
= \frac{\frac{1}{4 \pi} \cdot \sqrt{\frac{m \cdot d}{I_s}} \cdot 
\left(\sqrt{g + \Delta g} + \sqrt{g - \Delta g}\right)}
{\frac{1}{2 \pi} \cdot \sqrt{\frac{m \cdot g \cdot d}{I_s}}} \]
\[ \frac{f_2}{f_1} = \frac{\sqrt{g + \Delta g} + \sqrt{g - \Delta g}}
{2 \cdot \sqrt{g}} \]
Die Funktion von $\sqrt{x}$ ist konkav, bzw. rechtsgekrümmt. Daher hat eine
Änderung von $x$ in positiver Richtung einen kleineren Einfluss als eine gleich 
grosse Änderung in negativer Richtung. \\
$\rightarrow$ Die Uhr im Fahrstuhl läuft langsamer als eine nicht bewegte Uhr.\\
$\rightarrow$ Sam Servish erhält zu wenig, wenn er im Stundenlohn angestellt 
ist. 

\section{Lösung mit konkretem Beispiel}
Für das folgende Zahlenbeispiel werden folgende Annahmen getroffen: 
\begin{itemize}
  \item $g = 9.81 \frac{m}{s^2}$ 
  \item $\Delta g = 1 \frac{m}{s^2}$ 
  \item Der Fahrstuhl wird zu jedem Zeitpunkt entweder beschleunigt oder 
        abgebremst. 
\end{itemize}
\[ \frac{f_2}{f_1} = \frac{\sqrt{g + \Delta g} + \sqrt{g - \Delta g}}
{2 \cdot \sqrt{g}} = \frac{\sqrt{9.81\frac{m}{s^2} + 1 \frac{m}{s^2}} 
+ \sqrt{9.81\frac{m}{s^2} - 1 \frac{m}{s^2}}}
{2 \cdot \sqrt{9.81 \frac{m}{s^2}}} = \underline{\underline{0.998697}} \]
Bei einer Beschleunigung von $\pm 1 \frac{m}{s^2}$ läuft die Uhr im Fahrstuhl 
gegenüber einer nicht bewegten Uhr um 0.13 \% langsamer. 

\end{document}
