%coding:utf-8
\documentclass[a4paper, 10pt, fleqn]{article}

\usepackage{layout}

\title{Lösung Servish Rätsel 4 -- Wie der Zufall so spielt!}
\author{daniw}

\begin{document}
\maketitle
\newpage
\tableofcontents
\newpage

\section{Aufgabenstellung}

\subsection*{SERVISH RÄTSEL IV - Wie der Zufall so spielt!}

\subsubsection*{Turniermodus}
Der Gewinner erhält 50\% des Betrages, der Zweitbeste 30\% und der drittbeste 20\%. 

\subsubsection*{Bewerbung möglich bis}
26. Februar 2014, 
18:00 Uhr 

\subsubsection*{Rücktritt möglich bis}
28. Februar 2014, 
18:00 Uhr 

\subsubsection*{Projektabschluss}
02. März 2014, 
18:00 Uhr 

\subsubsection*{Rubrik}
Lifestyle \& Unterhaltung / Andere

\subsubsection*{Zusammenfassung}
Ich bin gespannt auf die kreativen Lösungen zum Rätsel Nummer IV

\subsubsection*{Beschreibung}
Sam Servish versucht, die Freizeit mit einem einfachen Zufallsexperiment 
erträglicher zu gestalten. Dabei wählt er zufällig reelle Zahlen zwischen 0 
und 1 aus. Er kann das tatsächlich!
\\\\
Zudem notiert er sich, wie viele Zahlen gezogen werden müssen, damit die Summe 
grösser als 1 wird. Er wiederholt dieses Experiment unzählige Male und 
errechnet den Mittelwert der Anzahl erforderlichen Ziehungen.
\\\\
Er staunt nicht schlecht, als er das Resultat erblickt!
\\\\
Wie gross ist dieser Mittelwert?
\\\\
------------------------------------------------\\
Sam Servish: "Nicht nur der Mittelwert interessiert mich, sondern auch der Weg 
dorthin"

\subsubsection*{Projektowner}
Sam Servish 

\subsubsection*{Eingetragen am}
13. Februar 2014 \\

\newpage

\section{Lösung}

\clearpage

\begin{appendices}
  \pagenumbering{roman}

  \section{Listing}
%   \lstinputlisting{../source/raetsel-4.py}

\end{appendices}

\end{document}
