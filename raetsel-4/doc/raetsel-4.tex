%coding:utf-8
\documentclass[a4paper, 10pt, fleqn]{article}

\usepackage{layout}

\title{Lösung Servish Rätsel 4 -- Wie der Zufall so spielt!}
\author{daniw}

\begin{document}
\maketitle
\newpage
\tableofcontents
\newpage

\section{Aufgabenstellung}

\subsection*{SERVISH RÄTSEL IV - Wie der Zufall so spielt!}

\subsubsection*{Turniermodus}
Der Gewinner erhält 50\% des Betrages, der Zweitbeste 30\% und der drittbeste 20\%. 

\subsubsection*{Bewerbung möglich bis}
26. Februar 2014, 
18:00 Uhr 

\subsubsection*{Rücktritt möglich bis}
28. Februar 2014, 
18:00 Uhr 

\subsubsection*{Projektabschluss}
02. März 2014, 
18:00 Uhr 

\subsubsection*{Rubrik}
Lifestyle \& Unterhaltung / Andere

\subsubsection*{Zusammenfassung}
Ich bin gespannt auf die kreativen Lösungen zum Rätsel Nummer IV

\subsubsection*{Beschreibung}
Sam Servish versucht, die Freizeit mit einem einfachen Zufallsexperiment 
erträglicher zu gestalten. Dabei wählt er zufällig reelle Zahlen zwischen 0 
und 1 aus. Er kann das tatsächlich!
\\\\
Zudem notiert er sich, wie viele Zahlen gezogen werden müssen, damit die Summe 
grösser als 1 wird. Er wiederholt dieses Experiment unzählige Male und 
errechnet den Mittelwert der Anzahl erforderlichen Ziehungen.
\\\\
Er staunt nicht schlecht, als er das Resultat erblickt!
\\\\
Wie gross ist dieser Mittelwert?
\\\\
------------------------------------------------\\
Sam Servish: "Nicht nur der Mittelwert interessiert mich, sondern auch der Weg 
dorthin"

\subsubsection*{Projektowner}
Sam Servish 

\subsubsection*{Eingetragen am}
13. Februar 2014 \\

\newpage

\section{Lösung}
Als erstes kann die Lösung gesucht werden, indem das Zufallsexperiment direkt 
wiederholt wird. Dabei werden Zufallszahlen zwischen 0 und 1 erzeugt. Diese 
werden so lange aufaddiert, bis die Summe grösser als 1 ist. Dabei werden die 
Anzahl dafür notwendiger Versuche gezählt. Das wird sehr oft wiederholt. 
Anschliessend wird der Mittelwert der benötigten Versuche berechnet. 
\lstinputlisting{../source/raetsel-4.py}
Die Rückgabe des obigen Programms ist $2.7182882564$. Dies entspricht 
näherungsweise der Eulerschen Zahl $e$. Doch wie kommt das? 

\newpage
\subsection{Theoretische Betrachtung}
Da jede Zahl einen statistisch unabhängigen Versuch darstellt, kann sie als 
uniform verteilt betrachtet werden. Daher ist auch die Summe der Zahlen 
uniform verteilt. Die Wahrscheinlichkeit, dass eine Uniform zwischen $a$ und 
$b$ verteile Zahl $z$ kleiner als die Quantile $q$ ist, kann wie folgt
beschrieben werden:  
\[ P(z < q) = \frac{q - a}{b - a} \]
In obiger Gleichung können nun die beiden Grenzen ($0$ und $n$) und die 
Quantile ($1$) eingesetzt werden. 
\[ P(z < 1) = \frac{1 - 0}{n - 0} = \frac{1}{n} \]
Das entspricht der Wahrscheinlichkeit, dass noch eine Zahl gezogen werden muss. 
Da bei Versuchen, die bereits grösser als $1$ sind keine weiteren Zahlen mehr 
gezogen werden, beträgt die Wahrscheinlichkeit, dass die Summe $s$ beim n-ten 
Ziehen einer Zahl grösser als 1 wird das Produkt aus den jeweiligen 
Wahrscheinlichkeiten bis n. 
\[ P =  \frac{1}{1} \cdot \frac{1}{2} \cdot \frac{1}{3} \cdot \frac{1}{4} \cdot 
\ldots \cdot \frac{1}{n} 
= \frac{1}{1 \cdot 2 \cdot 3 \cdot 4 \cdot \ldots \cdot n} = \frac{1}{n!} \]
Nun müssen die Wahrscheinlichkeiten für jede Anzahl Versuche aufsummiert 
werden. 
\[ \frac{1}{0!} + \frac{1}{1!} + \frac{1}{2!} + \frac{1}{3!} + \ldots 
+ \frac{1}{n!} = \sum\limits_{0}^{\infty} \left( \frac{1}{n!} \right) \]
Das ist die Definition von $e$. 
\[ \sum\limits_{0}^{\infty} \left( \frac{1}{n!} \right) = e \]

% \clearpage

% \begin{appendices}
%   \pagenumbering{roman}

%   \section{Listing}
%   \lstinputlisting{../source/raetsel-4.py}

% \end{appendices}

\end{document}
